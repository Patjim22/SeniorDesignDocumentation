\subsection{Boot Issues}
\begin{itemize}
    \item System takes a long time to boot
    \begin{itemize}
        \item Check that the SD card and power connector are properly seated.
        \item Check if the SD card has a slow read speed -- replace if so.
    \end{itemize}
    \item Error in Raspberry Pi boot process
    \begin{itemize}
        \item Check if the SD card is corrupted. Replace if it is.
        \item Flash SD card with a known working image.
    \end{itemize}
    \item System does not display UI fullscreen
    \begin{itemize}
        \item Change the sleep time in i3-config (inside the build directory) to be longer and re-flash
    \end{itemize}
\end{itemize}

\subsection{Communication Issues}
\begin{itemize}
    \item System does not connect to USB/Ethernet/etc.
    \begin{itemize}
        \item Re-seat all internal connectors.
        \item Verify cords are fully functional.
        \item Replace the Raspberry Pi.
    \end{itemize}
    \item System is not communicating with API.
    \begin{itemize}
        \item Check the system network connection.
        \item SSH into the system and run \bashline{getmac} to confirm the address was properly copied.
        \item Check for rejected communications in the admin interface, replace any incorrectly entered MAC address.
    \end{itemize}
    \item Admin Interface is not communicating with database.
    \begin{itemize}
        \item Check if the database is live.
        \item Log into the database and check for rejected connections.
        \item Update database user certifications.
    \end{itemize}
\end{itemize}

\subsection{Image Issues}
\begin{itemize}
    \item Image needs superuser permission to build/update/flash.
    \begin{itemize}
        \item Use a system where you have superuser access.
    \end{itemize}
    \item Image is corrupted on SD card.
    \begin{itemize}
        \item Mount the SD card data partition and run \bashline{fsck -fy FILESYSTEM}
        \item Decompress the latest image, loopback mount the data partition, and run \bashline{fsck -fy FILESYSTEM}
        \item If the image has no issues, but the SD card is corrupted after writing it again, discard the SD card.
        \item If the image has issues, go to an older image and check it. If it works, update the older image and reflash.
        \item If no older issues work, rebuild an image from scratch.
    \end{itemize}
\end{itemize}